
\usepackage[T1]{fontenc} %%%key to get copy and paste for the code!

\usepackage{amssymb,amsmath}

% Source Code
\usepackage{color}
\usepackage{textcomp}
\usepackage{listings}
\usepackage{amsfonts}
\usepackage{courier}

\definecolor{source}{gray}{0.9}

% my comment style
\newcommand{\myCommentStyle}[1]{{\ttfamily\footnotesize\color{gray!100!white} #1}}

% my string style
\newcommand{\myStringStyle}[1]{{\ttfamily\footnotesize\color{violet!100!black} #1}}

% my symbol style
\newcommand{\mySymbolStyle}[1]{{\ttfamily\footnotesize\color{violet!100!black} #1}}

% my keyword style
\newcommand{\myKeywordStyle}[1]{{\ttfamily\footnotesize\color{green!70!black} #1}}

% my global style
\newcommand{\myGlobalStyle}[1]{{\ttfamily\footnotesize\color{blue!100!black} #1}}

% my number style
\newcommand{\myNumberStyle}[1]{{\ttfamily\footnotesize\color{brown!100!black} #1}}

\lstset{
language={},
% characters
tabsize=3,
escapechar={!},
keepspaces=true,
breaklines=true,
alsoletter={\#},
literate={\$}{{{\$}}}1,
breakautoindent=true,
columns=fullflexible,
showstringspaces=false,
% background
frame=single,
aboveskip=1em, % automatic space before
framerule=0pt,
basicstyle=\ttfamily\footnotesize\color{black},
keywordstyle=\myKeywordStyle,% keyword style
commentstyle=\myCommentStyle,% comment style
frame=single,%
backgroundcolor=\color{source},
% numbering
stepnumber=1,
numbersep=10pt,
numberstyle=\tiny,
numberfirstline=true,
% caption
captionpos=b,
% formatting (html)
moredelim=[is][\bfseries]{<b>}{</b>},
moredelim=[is][\textit]{<i>}{</i>},
moredelim=[is][\underbar]{<u>}{</u>},
moredelim=[is][\color{red}\uwave]{<wave>}{</wave>},
moredelim=[is][\color{red}\sout]{<del>}{</del>},
moredelim=[is][\color{blue}\underbar]{<ins>}{</ins>},
% smalltalk stuff
morecomment=[s][\myCommentStyle]{"}{"},
%    morecomment=[s][\myvs]{|}{|},
morestring=[b][\myStringStyle]',
moredelim=[is][]{<sel>}{</sel>},
moredelim=[is][]{<rcv>}{</rcv>},
moredelim=[is][\itshape]{<symb>}{</symb>},
moredelim=[is][\scshape]{<class>}{</class>},
morekeywords={true,false,nil,self,super,thisContext},
identifierstyle=\idstyle,
}

\makeatletter
\newcommand*\idstyle[1]{%
\expandafter\id@style\the\lst@token{#1}\relax%
}
\def\id@style#1#2\relax{%
\ifnum\pdfstrcmp{#1}{\#}=0%
% this is a symbol
\mySymbolStyle{\the\lst@token}%
\else%
\edef\tempa{\uccode`#1}%
\edef\tempb{`#1}%
\ifnum\tempa=\tempb%
% this is a global
\myGlobalStyle{\the\lst@token}%
\else%
\the\lst@token%
\fi%
\fi%
}
\makeatother


%\newcommand{\ct}{\lstinline[backgroundcolor=\color{white}]}
%\newcommand{\needlines}[1]{\Needspace{#1\baselineskip}}
\newcommand{\lct}{\texttt}

\lstnewenvironment{code}{%
    \lstset{%
    % frame=lines,
    frame=single,
    framerule=0pt,
    mathescape=false
    }%
    \noindent%
    \minipage{\linewidth}%
}{%
    \endminipage%
}%


\lstnewenvironment{codeWithLineNumbers}{%
    \lstset{%
    % frame=lines,
    frame=single,
    framerule=0pt,
    mathescape=false,
    numbers=left
    }%
    \noindent%
    \minipage{\linewidth}%
}{%
    \endminipage%
}%

%For simple inlined code
\newcommand{\ct}{\texttt}

%utiles e.g., i.e., c.f.
\usepackage{xspace}
\newcommand{\eg}{\emph{e.g.,}\xspace}
\newcommand{\ie}{\emph{i.e.,}\xspace}
\newcommand{\cf}{\emph{c.f.}\xspace}

%For comments
\usepackage{ifthen}
\newboolean{showcomments}
\setboolean{showcomments}{true}
\ifthenelse{\boolean{showcomments}}
  {\newcommand{\bnote}[2]{
	\fbox{\bfseries\sffamily\scriptsize#1}
    {\sf\small$\blacktriangleright$\textit{#2}$\blacktriangleleft$}
    % \marginpar{\fbox{\bfseries\sffamily#1}}
   }
   \newcommand{\cvsversion}{\emph{\scriptsize$-$Id: macros.tex,v 1.1.1.1 2007/02/28 13:43:36 bergel Exp $-$}}
	\newcommand{\del}[1]{\bnote{Remove}{\textcolor{gray} #1}}
  }
  {\newcommand{\bnote}[2]{}
   \newcommand{\cvsversion}{}
	\newcommand{\del}[1]{}
  } 

\newcommand{\gp}[1]{\bnote{Guille}{#1}}
\newcommand{\fd}[1]{\bnote{Fer}{#1}}

\newtheorem{definition}{Definición}

\newenvironment{codeNonSmalltalk}
{\begin{alltt}\ttfamily}
{\end{alltt}\normalsize}

% Para insertar boxes
\usepackage[framemethod=TikZ]{mdframed}
\mdfdefinestyle{BoxFrame}{%
    linecolor=black,
    outerlinewidth=1pt,
    roundcorner=20pt,
    innertopmargin=\baselineskip,
    innerbottommargin=\baselineskip,
    innerrightmargin=20pt,
    innerleftmargin=20pt,
    backgroundcolor=gray!30!white}

\renewcommand{\labelitemi}{$\bullet$}
\renewcommand{\labelitemii}{$\cdot$}
\renewcommand{\labelitemiii}{$\diamond$}
\renewcommand{\labelitemiv}{$\ast$}