Juego de rol con manejo de equipo (Baldur’s Gate)

La idea seria programar algo similar a lo que seria baldur’s gate, (un juego actual similar seria, mas bien, el dragon age). 
En estos juegos uno tiene un personaje propio, pero ademas maneja un grupo de personajes. 

El estilo en general esta basado en dungeon and dragons. Este estilo de juego, a la hora de programar 
es realmente muy complejo, siendo que cada personaje tiene una raza (humano, elfo, halfling, etc), 
una o mas clases (ladrón, paladin, hechicero, mago, etc) con clases prestige (clases de especialización). 
Y esto es antes de poder siquiera pensar en los atributos psicofisicos y/o el transfondo del personaje . 

Lo que propongo entonces seria pensar en un juego donde uno puede tener un grupo de personajes. 
Y este grupo de personajes tendría que ser capaz de completar una misión. (por ejemplo, pelearse con otro grupo) 

Pero empecemos desde el principio
Un personaje en este tipo de juego tiene que tener vida y alguna forma de energía utilizable para los poderes
especiales. De entrada estas barras se pueden llenar a través de alimentos y/o pociones. Creo que no tiene sentido
pensar en diferentes tipos de energía, ya que con un tipo es suficiente para llegar a comprender el concepto detrás.
Pero en caso de que se crea necesario, puede haber pociones de diferentes tipos, para diferentes tipos de energía.
Pociones de magia, aguante (stamina), vida, etc.  

Entonces acá ya tenemos una pepita que come y toma objetos que le dan vida y energía.

Una vez acá, podemos empezar a pensar como modelar el personaje en términos de su rol en el juego: podemos
usar el sistema de clases de dungeon and dragons, que es estricto “el mago hace a b y c” o algún modelo mas dinámico,
donde cada personaje sea mas bien personalizable y uno puede darle habilidades puntuales como “uso de armas blancas”
“casteo de magia” “uso de magia” “uso de habilidades acrobáticas” etc.

Creo que para PDEP, el sistema de clases va a ser lo mas interesante, mientras que para diseño/old tadp,
sin dudas el modelo dinámico tiene mucho a explotar. 

Una vez tengamos el primer modelo de personaje usable, por ejemplo un mago, podemos equiparlo con una varita mágica
y hacerlo atacar otro personaje… a distancia (acá tendríamos el uso de energía y también el modelo age of empires
de ataque). Ahora podemos agregar otro modelo de personaje, por ejemplo un guerrero, a quien no pueda equiparse
con una varita tal vez. Y quien pueda usar armadura (para disminuir el daño) y espada, para atacar cuerpo a cuerpo. 

De la misma manera podemos agregar otros modelos de personaje. pero hasta acá creo que con dos alcanza.
Una vez tengamos dos tipos de personajes, podemos tener nuestro equipo. Cuando tenemos un equipo tenemos
dos tipos de formas de interactuar con nuestros personajes: individualmente o como equipo. En el manejo a
nivel de equipo entraría sin problemas un buen polimorfismo y colecciones. 

Como bonus track para diseño/old tadp se podría pensar en nivel estratégico, por ejemplo, cuando uno pide que
el grupo se mueva de un lado a otro, que se muevan en cierta formación, o que frente a ciertas circunstancias
(vida al 20\% ponele) hagan algo (tomar una poción de vida o comerse un pancho).

Una vez que tengamos esto podemos tener dos grupos y hacer que se peleen. Acá podemos hacerlo fácil para pdep, 
pensando tal vez en que el grupo pelee con un monstruo grande (para lo que necesitaríamos un monstruo), 
para diseño/tadp podemos pensar en estrategias para que un grupo ataque al otro, por ejemplo, matar primero al
hechicero enemigo que mas daño hace es siempre una gran estrategia. Que se peleen si no seleccionando como target
el enemigo contra el que tengan mas ventaja. 

Una vez que tengamos la pelea entre dos grupos, creo que ya explotamos lo suficiente este aspecto y podemos pasar
a la idea de misión. Por ejemplo, recuperar una gema de un grupo de ladrones. Para esto haría falta agregarle
una mochila a cada personaje. Sin duda acá podemos meter el uso de observer o similar. 

Otros tipos de misiones habilitarían demandar un análisis de estado de cada personaje, de ahí puede salir un observer. 

- Este ejemplo se puede mantener simple, a nivel pepita ++ y se puede complicar hasta donde se quiera (por ejemplo, 
se puede agregar la ‘tirada de dados’ para calcular la probabilidad/cantidad de daño, se puede agregar comercio, 
etc). Hay que tener en cuenta que cada agregado puede complicar las cosas de manera exponencial (por ejemplo, no se
si me atrevería a poner items especiales que hagan otra cosa que no sea reducir el daño o aumentar el máximo de
vida/energia, así como tampoco a modelar hechizos de penalización o bonificación) 

- A través del concepto de misión se puede proponer una suite de tests que hagan que se pueda comprobar a
cierto nivel que un ejercicio este bien hecho. (lo que lo haría compatible con la idea general de aprendizaje
menos-supervisado).

- Pertenece al género video juegos (compatible con la idea de usar wollok games) , lo que si, de un lado que
podría ser un poco menos casual, por lo que tal vez tendría menos llegada al lado que cada uno de considera
potencialmente ‘divertido’. 
